\PassOptionsToPackage{unicode=true}{hyperref} % options for packages loaded elsewhere
\PassOptionsToPackage{hyphens}{url}
%
\documentclass[]{article}
\usepackage{lmodern}
\usepackage{amssymb,amsmath}
\usepackage{ifxetex,ifluatex}
\usepackage{fixltx2e} % provides \textsubscript
\ifnum 0\ifxetex 1\fi\ifluatex 1\fi=0 % if pdftex
  \usepackage[T1]{fontenc}
  \usepackage[utf8]{inputenc}
  \usepackage{textcomp} % provides euro and other symbols
\else % if luatex or xelatex
  \usepackage{unicode-math}
  \defaultfontfeatures{Ligatures=TeX,Scale=MatchLowercase}
\fi
% use upquote if available, for straight quotes in verbatim environments
\IfFileExists{upquote.sty}{\usepackage{upquote}}{}
% use microtype if available
\IfFileExists{microtype.sty}{%
\usepackage[]{microtype}
\UseMicrotypeSet[protrusion]{basicmath} % disable protrusion for tt fonts
}{}
\IfFileExists{parskip.sty}{%
\usepackage{parskip}
}{% else
\setlength{\parindent}{0pt}
\setlength{\parskip}{6pt plus 2pt minus 1pt}
}
\usepackage{hyperref}
\hypersetup{
            pdftitle={Analysing Simulation Study Results},
            pdfborder={0 0 0},
            breaklinks=true}
\urlstyle{same}  % don't use monospace font for urls
\usepackage[margin=1in]{geometry}
\usepackage{color}
\usepackage{fancyvrb}
\newcommand{\VerbBar}{|}
\newcommand{\VERB}{\Verb[commandchars=\\\{\}]}
\DefineVerbatimEnvironment{Highlighting}{Verbatim}{commandchars=\\\{\}}
% Add ',fontsize=\small' for more characters per line
\usepackage{framed}
\definecolor{shadecolor}{RGB}{248,248,248}
\newenvironment{Shaded}{\begin{snugshade}}{\end{snugshade}}
\newcommand{\AlertTok}[1]{\textcolor[rgb]{0.94,0.16,0.16}{#1}}
\newcommand{\AnnotationTok}[1]{\textcolor[rgb]{0.56,0.35,0.01}{\textbf{\textit{#1}}}}
\newcommand{\AttributeTok}[1]{\textcolor[rgb]{0.77,0.63,0.00}{#1}}
\newcommand{\BaseNTok}[1]{\textcolor[rgb]{0.00,0.00,0.81}{#1}}
\newcommand{\BuiltInTok}[1]{#1}
\newcommand{\CharTok}[1]{\textcolor[rgb]{0.31,0.60,0.02}{#1}}
\newcommand{\CommentTok}[1]{\textcolor[rgb]{0.56,0.35,0.01}{\textit{#1}}}
\newcommand{\CommentVarTok}[1]{\textcolor[rgb]{0.56,0.35,0.01}{\textbf{\textit{#1}}}}
\newcommand{\ConstantTok}[1]{\textcolor[rgb]{0.00,0.00,0.00}{#1}}
\newcommand{\ControlFlowTok}[1]{\textcolor[rgb]{0.13,0.29,0.53}{\textbf{#1}}}
\newcommand{\DataTypeTok}[1]{\textcolor[rgb]{0.13,0.29,0.53}{#1}}
\newcommand{\DecValTok}[1]{\textcolor[rgb]{0.00,0.00,0.81}{#1}}
\newcommand{\DocumentationTok}[1]{\textcolor[rgb]{0.56,0.35,0.01}{\textbf{\textit{#1}}}}
\newcommand{\ErrorTok}[1]{\textcolor[rgb]{0.64,0.00,0.00}{\textbf{#1}}}
\newcommand{\ExtensionTok}[1]{#1}
\newcommand{\FloatTok}[1]{\textcolor[rgb]{0.00,0.00,0.81}{#1}}
\newcommand{\FunctionTok}[1]{\textcolor[rgb]{0.00,0.00,0.00}{#1}}
\newcommand{\ImportTok}[1]{#1}
\newcommand{\InformationTok}[1]{\textcolor[rgb]{0.56,0.35,0.01}{\textbf{\textit{#1}}}}
\newcommand{\KeywordTok}[1]{\textcolor[rgb]{0.13,0.29,0.53}{\textbf{#1}}}
\newcommand{\NormalTok}[1]{#1}
\newcommand{\OperatorTok}[1]{\textcolor[rgb]{0.81,0.36,0.00}{\textbf{#1}}}
\newcommand{\OtherTok}[1]{\textcolor[rgb]{0.56,0.35,0.01}{#1}}
\newcommand{\PreprocessorTok}[1]{\textcolor[rgb]{0.56,0.35,0.01}{\textit{#1}}}
\newcommand{\RegionMarkerTok}[1]{#1}
\newcommand{\SpecialCharTok}[1]{\textcolor[rgb]{0.00,0.00,0.00}{#1}}
\newcommand{\SpecialStringTok}[1]{\textcolor[rgb]{0.31,0.60,0.02}{#1}}
\newcommand{\StringTok}[1]{\textcolor[rgb]{0.31,0.60,0.02}{#1}}
\newcommand{\VariableTok}[1]{\textcolor[rgb]{0.00,0.00,0.00}{#1}}
\newcommand{\VerbatimStringTok}[1]{\textcolor[rgb]{0.31,0.60,0.02}{#1}}
\newcommand{\WarningTok}[1]{\textcolor[rgb]{0.56,0.35,0.01}{\textbf{\textit{#1}}}}
\usepackage{graphicx,grffile}
\makeatletter
\def\maxwidth{\ifdim\Gin@nat@width>\linewidth\linewidth\else\Gin@nat@width\fi}
\def\maxheight{\ifdim\Gin@nat@height>\textheight\textheight\else\Gin@nat@height\fi}
\makeatother
% Scale images if necessary, so that they will not overflow the page
% margins by default, and it is still possible to overwrite the defaults
% using explicit options in \includegraphics[width, height, ...]{}
\setkeys{Gin}{width=\maxwidth,height=\maxheight,keepaspectratio}
\setlength{\emergencystretch}{3em}  % prevent overfull lines
\providecommand{\tightlist}{%
  \setlength{\itemsep}{0pt}\setlength{\parskip}{0pt}}
\setcounter{secnumdepth}{0}
% Redefines (sub)paragraphs to behave more like sections
\ifx\paragraph\undefined\else
\let\oldparagraph\paragraph
\renewcommand{\paragraph}[1]{\oldparagraph{#1}\mbox{}}
\fi
\ifx\subparagraph\undefined\else
\let\oldsubparagraph\subparagraph
\renewcommand{\subparagraph}[1]{\oldsubparagraph{#1}\mbox{}}
\fi

% set default figure placement to htbp
\makeatletter
\def\fps@figure{htbp}
\makeatother


\title{Analysing Simulation Study Results}
\author{}
\date{\vspace{-2.5em}}

\begin{document}
\maketitle

\begin{center}\rule{0.5\linewidth}{0.5pt}\end{center}

\hypertarget{necessary-libraries}{%
\subsubsection{Necessary Libraries}\label{necessary-libraries}}

\begin{Shaded}
\begin{Highlighting}[]
\KeywordTok{library}\NormalTok{(ggplot2)}
\KeywordTok{library}\NormalTok{(dplyr)}
\end{Highlighting}
\end{Shaded}

\begin{verbatim}
## 
## Attaching package: 'dplyr'
\end{verbatim}

\begin{verbatim}
## The following objects are masked from 'package:stats':
## 
##     filter, lag
\end{verbatim}

\begin{verbatim}
## The following objects are masked from 'package:base':
## 
##     intersect, setdiff, setequal, union
\end{verbatim}

\begin{Shaded}
\begin{Highlighting}[]
\KeywordTok{library}\NormalTok{(RColorBrewer)}
\end{Highlighting}
\end{Shaded}

\hypertarget{read-in-results}{%
\subsubsection{Read in results}\label{read-in-results}}

\begin{Shaded}
\begin{Highlighting}[]
\NormalTok{model_performance <-}\StringTok{ }\KeywordTok{readRDS}\NormalTok{(}\StringTok{"imputation_results.rds"}\NormalTok{)}
\NormalTok{mean_diff_estimates <-}\StringTok{ }\KeywordTok{readRDS}\NormalTok{(}\StringTok{"imputation_results_2.rds"}\NormalTok{)}
\end{Highlighting}
\end{Shaded}

\begin{center}\rule{0.5\linewidth}{0.5pt}\end{center}

\hypertarget{model-performance}{%
\subsection{Model Performance}\label{model-performance}}

\hypertarget{take-a-look-at-the-model-performances-output}{%
\paragraph{Take a look at the model performances
output}\label{take-a-look-at-the-model-performances-output}}

A snapshot of this dataframe can be seen below

\begin{Shaded}
\begin{Highlighting}[]
\KeywordTok{head}\NormalTok{(model_performance, }\DecValTok{15}\NormalTok{)}
\end{Highlighting}
\end{Shaded}

\begin{verbatim}
##    performance performance_type columns  model type % drop
## 1    1.0000169            nrmse   cont9 simple cont    0.3
## 2    1.3945263            nrmse   cont9   mice cont    0.3
## 3    0.9841713            nrmse   cont9 forest cont    0.3
## 4    1.0000046            nrmse   cont5 simple cont    0.3
## 5    1.4108656            nrmse   cont5   mice cont    0.3
## 6    0.9990116            nrmse   cont5 forest cont    0.3
## 7    1.0002179            nrmse   cont9 simple cont    0.2
## 8    1.3902305            nrmse   cont9   mice cont    0.2
## 9    0.9822623            nrmse   cont9 forest cont    0.2
## 10   1.0000021            nrmse   cont5 simple cont    0.2
## 11   1.4002173            nrmse   cont5   mice cont    0.2
## 12   0.9908740            nrmse   cont5 forest cont    0.2
## 13   1.0000000            nrmse   cont9 simple cont    0.1
## 14   1.4196324            nrmse   cont9   mice cont    0.1
## 15   0.9723497            nrmse   cont9 forest cont    0.1
\end{verbatim}

The measures of model performance chosen have been the:

\begin{itemize}
\item
  \textbf{Normalised Root Mean Squared Error} for the \emph{Continuous
  Features}.

  Recall, the formula for the NRMSE is
  \(\sqrt{mean((X_{true} − X_{imp})^2)/var(X_{true})}\), where
  \(X_{true}\) the complete data matrix and \(X_{imp}\) the imputed data
  matrix.

  In simple words, this metric is made up of the squared error over the
  deviation of the true underlying data, which normalises the error term
  and allows for meaningful comparisons across the different column
  estimates and scenarios. Moreover, for good performance we expect
  values to be close to zero.
\end{itemize}

\includegraphics{MICE-Imputation-Results_files/figure-latex/unnamed-chunk-4-1.pdf}

\begin{itemize}
\tightlist
\item
  \textbf{Accuracy} for the \emph{Categorical Features}
\end{itemize}

\includegraphics{MICE-Imputation-Results_files/figure-latex/unnamed-chunk-5-1.pdf}

\begin{center}\rule{0.5\linewidth}{0.5pt}\end{center}

\hypertarget{mean-difference-estimates}{%
\subsection{Mean Difference Estimates}\label{mean-difference-estimates}}

\hypertarget{take-a-look-at-the-estimated-output}{%
\paragraph{Take a look at the estimated
output}\label{take-a-look-at-the-estimated-output}}

A snapshot of this dataframe can be seen below

\begin{Shaded}
\begin{Highlighting}[]
\KeywordTok{head}\NormalTok{(mean_diff_estimates, }\DecValTok{15}\NormalTok{)}
\end{Highlighting}
\end{Shaded}

\begin{verbatim}
##        mean_diff  model % drop
## 1  -0.0011299024 simple    0.3
## 2   0.0022754617   mice    0.3
## 3   0.0151230981   mice    0.3
## 4  -0.0028426041   mice    0.3
## 5   0.0023161684 forest    0.3
## 6   0.0007328600 simple    0.3
## 7   0.0008139346   mice    0.3
## 8   0.0012448486   mice    0.3
## 9   0.0045700727   mice    0.3
## 10  0.0071474805 forest    0.3
## 11  0.0040413478 simple    0.2
## 12  0.0257665885   mice    0.2
## 13 -0.0126167648   mice    0.2
## 14  0.0045644662   mice    0.2
## 15 -0.0112796256 forest    0.2
\end{verbatim}

\hypertarget{mean-difference-between-the-true-values-and-the-imputed-values}{%
\subsubsection{1. Mean Difference between the True Values and the
Imputed
Values}\label{mean-difference-between-the-true-values-and-the-imputed-values}}

\hypertarget{a-visualisations-and-initial-exploration}{%
\paragraph{a) Visualisations and Initial
Exploration}\label{a-visualisations-and-initial-exploration}}

\hypertarget{what-do-i-mean-by-mean-difference}{%
\subparagraph{What do I mean by mean
difference?}\label{what-do-i-mean-by-mean-difference}}

A good way to assess model performance in this instance, is to take a
look at the \textbf{ability of each model to correctly impute the
missing values}. This measure is obtained by taking the \textbf{average
of the differences} between all the imputed values
\(\bar{x}_{\text{imp}, i}\) for column \(i\) and all the true values
\(\bar{x}_{\text{true}, i}\) for column \(i\).

Hence:
\(MD_i = mean(\bar{x}_{\text{imp}, i} - \bar{x}_{\text{true}, i})\)

This measure is computed for each column of interest, where the column i
was randomly chosen in the simulation as the column to drop values from,
and for each scenario combination, where the scenario combinations are
\(\text{\{cont_only, 0.1\}}\), \(\text{\{cat_only, 0.1\}}\),
\(\text{\{both, 0.1\}}\), \(\text{\{cont_only, 0.2\}}\), \ldots{}.

\hypertarget{how-do-i-make-use-of-these-measures}{%
\subparagraph{How do I make use of these
measures?}\label{how-do-i-make-use-of-these-measures}}

First, one may wish to take a look at a group comparison. This can help
highlight better performing models against poorer performing models in
one go.

\begin{itemize}
\tightlist
\item
  For \emph{poor performance} of a model, I expect to see the
  distribution of \(\bar{D}\) with an overall \textbf{mean further away
  from zero and a high variability}.
\item
  For \emph{good performance} of a model, I expect to see the
  distribution of \(\bar{D}\) with an overall \textbf{mean closer to
  zero and a smaller variability}.
\end{itemize}

Therefore;

\includegraphics{MICE-Imputation-Results_files/figure-latex/unnamed-chunk-7-1.pdf}

\end{document}
